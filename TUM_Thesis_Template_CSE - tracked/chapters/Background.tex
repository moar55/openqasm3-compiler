\chapter{Background}
% add figures
\label{chapter:Background}
\section{Heterogeneous Computing}
Heterogeneous computing refers to the use of multiple types of processing units
or cores, with the aim
of enhancing the overall performance and energy efficiency of 
the system.
When mixing different types of processing units, we are able to
utilize the strengths of each processing unit, while mitigating the weaknesses.
These processing units can include
central processing units (CPUs), but also other processing units - sometimes referred to as accelerators -
like  graphical processing units (GPUs), 
field programmable gate arrays (FPGAs), and application-specific integrated circuits(ASICs).
Each of these processing units, can perform  
specific types of computations with varying degrees of efficiency. 
For our purposes we will focus on heterogeneous computing that combines classical
computing using CPUs and quantum computing using quantum processing units (QPUs), quantum-classical
heterogeneous computing.
In this kind of heterogeneous computing we face the challenge of quantum
decoherence. Quantum decoherence is the phenomenon where quantum systems interact
with the environment, causing them to lose their
quantum behavior and fallback to a classical one.
These two challenges are different depending on whether we are applying
near-time or real-time quantum computing. Real-time quantum computing requires calculations to
complete within - or faster than - the qubit coherence time, while near-time quantum computing can tolerate high
latency \cite{Near-Real}.
% paraphrase
\section{Quantum Computing Programming}
Quantum computing programming languages are designed to help developers write
and execute quantum algorithms on quantum computers. Although quantum computers
are still in the early stages of development, there has been significant
progress in the development of quantum computing programming languages. In this
thesis we are only interested in OPENQASM 3.0. OPENQASM - short for Open quantum
assembly - is a low level, quantum circuit-like language for quantum computing.
With its newest edition - OPENQASM 3.0 - OPENQASM is no longer a strictly
quantum programming language, but rather a Heterogeneous language. It allows for
strictly quantum operations provided in OPENQASM 2.0, but also allows classical
operations such as control operations (i.e: for and while loops, if
statements...etc), extensive arithmetic operations, among many other operations.
This heterogeneity of the language, being an open standard, and the large
adoption by many scientists are all reasons for focusing on OPENQASM 3.0 in this
thesis.
\section{Compilation}
\subsection{Compilation Overview}
Compilation in computer science is the process of translating a source language
to a target language. Typically, this is done in an effort to translate a
high-level language, to a lower level language, or to machine code to be executed by
the computer. In this section we provide an overview of the compilation process,
and the different stages of compilation, based on a more detailed description by 
 \cite[p. 4-11]{Compilers}.
Normally, a compiler has two main parts, the frontend and the backend.
The frontend applies the analysis phase of a compiler, while the backend applies
the synthesis phase. The analysis phase, among possibly other steps, has three
main steps:
\begin{itemize}
  \item Lexical Analysis
  \item Syntax Analysis
  \item Semantic Analysis
\end{itemize}
Lexical analysis, also called tokenization, is the process of splitting the
input file into multiple tokens. If invalid tokens are found a tokenization
error is generated. Syntax analysis, also known as parsing, is the process of
verifying that a token stream is syntactically well-formed, and constructs a
syntax tree from a valid set of tokens. This essentially verifies that the input program
abides to our language's grammar syntax, and provides an intermediate
representation for the next stage of analysis. The program that performs the
parsing is called the parser. Finally, semantic analysis is the process of
verifying that our program is sound in regard to our language's semantics. For
example, in a strictly typed language an int variable can't be assigned a
string. During semantic analysis, the syntax tree can be annotated to provide
useful information (i.e: implicit type casting information) and produces what is
known as an abstract syntax tree.

The next stage, the synthesis phase, consists also of three main phases:
\begin{itemize}
  \item Intermediate Code Generation
  \item Program Optimization
  \item Code Generator
\end{itemize}
Intermediate code generation is the process of generating a low-level
intermediate representation (IR) code. This IR usually has the benefit of having
a small instruction set, and simple syntax and semantics. This makes the next stage, 
the program optimization stage, much simpler.
Program optimization is the process of transformation from a less efficient program 
to a more efficient one. This can be done by removing redundant instructions,
or instructions with more efficient ones. Intuitively, this stage is optional yet
very essential. It can be split into two stages, machine independent and machine
dependent optimization, where the former is done before target code generation,
and the latter is done after.
Finally, target code generation is the process of generating the
target code to be executed on the machine. %TODO: makes this more concise
\subsection{ANTLR}
ANTLR, which stands for Another Tool for Language Recognition, is a tool for generating parsers 
that can be utilized for different tasks relevant to components of the compilation process,
and text analysis and processing in general.
It is frequently utilized in developing programming languages, tools, and frameworks.
It's often
used in the creation of programming languages, tools, and frameworks \cite{ANTLR}. ANTLR generates a parser from a grammar defined in a domain
specific language (DSL), designed by ANTLR. ANTLR provides runtimes in multiple
programming languages. You can choose which runtime you want your parser to be
written in, and then you can construct and walk parse trees from input programs using 
ANTLR's runtime library.
\subsection{LLVM and MLIR}
\subsubsection{LLVM}
LLVM, which stands for Low Level Virtual Machine, is a compiler framework that
enables comprehensive program analysis and transformation for various programs
throughout their entire lifespan. It does this by providing high-level
information to compiler transformations at compile-time, link-time, run-time,
and in idle time between runs. LLVM uses a common, low-level code representation
in static single assignment (SSA) form, which includes a simple,
language-independent type-system that exposes the primitives commonly used to
implement high-level language features. Additionally, LLVM features an
instruction for typed address arithmetic and a simple mechanism that can be used
to implement the exception handling features of high-level languages and
\texttt{setjmp}/\texttt{longjmp} in C uniformly and efficiently\cite{LLVM}.
\subsection{MLIR}
Multi-level intermediate representation (MLIR) is a novel approach for creating
reusable and adaptable domain specific compilers. It is designed to provide an
extensible common IR of SSA form, that can be used to represent multiple Domain
specific languages (DSLs). This elevates the need for a new IR for each new DSL
language, and allows for the reuse of existing IRs, especially within the same
domain. Moreover, MLIR can be then lowered to LLVM IR, and utilize the LLVM
tool-chain providing multiple lower-level optimizations and the ability to target
different kinds of hardware accelerators with different architectures. The
structure of MLIR's IR has few key abstractions - types, attributes, and
operations. These main three abstractions can sufficiently extend the IR to its
users needs. There are two more abstractions, regions and blocks, but they are
extension of operations, where a block is a list of operations, and a region is
a list of blocks. A Dialect refers to a unique namespace that encompasses a
coherent set of operations, types, and attributes that are logically grouped
together. MLIR provides a lot of public dialects catering to different domains,
such as the \texttt{Arith} Dialect, the \texttt{Math} Dialect, the \texttt{MemRef} (Memory Reference) Dialect,
and the Vector Dialect...etc. One can also create their own dialect 
for their own domain specific application. MLIR also provides a set of tools to
help with the creation of new dialects, such as the MLIR \texttt{TableGen} tool, which
allows for the creation of new dialects by writing a simple DSL.
However, it should be noted that if one desires to lower down to LLVM IR, they
will have to lower their dialect's operations in terms of the dialects provided 
by MLIR (include the LLVM IR). There are multiple approaches and defined classes that can aid with that.
\subsection{Efforts to Leverage MLIR for heterogeneous quantum computing}
Similar work to the one in this thesis was done by a team at Oak Ridge National Laboratory.
QCOR, a tool created by this team, is a language extension specification for creating quantum kernels (in C++ and Python),
and a compiler platform for heterogeneous computing \cite{qcor-cpp}\cite{qcor-python}.
%TODO: maybe cite their paper?
To cater for other publicly available quantum DSL's, mainly OPENQASM 2.0 and OPENQASM 3.0, they created an 
MLIR dialect, Quantum Dialect, that allows for representation of the quantum operations in MLIR \cite{qcor-openqasm}\cite{qcor-openqasm-3.0}.
They also used MLIR's pattern rewriter to perform optimizations on the quantum IR.
Finally, they lowered it to LLVM IR in a specific format that can be executed by their own runtime, 
that can target different quantum simulators and quantum hardware, alongside classical hardware.
This effort was a great inspiration for this thesis,
where we similarly created an MLIR dialect for quantum operations, and 
also shared some similarities in the quantum operations and the structure of lowering 
to the \texttt{quantum} dialect.
However, there were also many areas that exhibited vast differences.
Ranging from major differences (additions and omissions) in the operations that were defined in the quantum dialect, 
as well as the differentiation between a generic quantum dialect and 
a more restricted one, the way quantum operations were optimized, the construction 
of a quantum simulator written purely in MLIR, and in general differences in the way we lowered OPENQASM 3.0 all the way down to LLVM IR.
 Most noticeably as mentionend earlier we used the latest MLIR version allowing us to incorporate
dialects like the \texttt{vector} dialect to be able to represent classical registers, for storing measurements or 
otherwise, which allows for vectorization and more efficient classical computation.
Moreover, we used the latest OPENQASM 3.0 grammar defined at the time of our development, which led to 
differences in the way we lowered to the \texttt{quantum} dialect.
Because of these differences in strategy, in the underlying library used and in OPENQASM 3.0 grammar,
this project is naturally a complete rewrite, but was definitely vastly inspired by QCOR.