% !TEX root = ../main.tex
\chapter{Introduction}
\label{chapter:Introduction}
Combination of quantum and classical computation as a form of Hetorgenous Computing allows us
to solve very interesting problems. Many quantum algorithms and applications relies
deeply on this combination. Whether near-time heterogenous computing (i.e: Variational Quantum Eigen Solver),
or real-time (i.e: quantum error correction algorithms). While near-time heterogenous computing
is already achievable with current technology, albeit on a small scale, 
real-time heterogenous computing is still a
challenge,
%TODO: fact check
due to the issue of decoherence time of quantum systems, and the requirement for
very low classical data transfer latency to account for this issue. In either
of these cases, we can use a quantum processing unit (QPU) as an accelerator
alongside a cpu, where the QPU can solve problems that QPU's excel in and the
CPU can solve problems that CPU's excel in, such as conditionals, loops...etc.
This allows us to apply quantum algorithms that uses a cpu as a co-processor,
but also to improve the performance of a a QPU as in the case of quantum error
correction.  This concept of heterogenousity can also be extended to include
other types of accelerators such as GPUs, FPGAs, and ASICs, to benefit from them
in domains where they excel. However, in this thesis we will focus on the
combination of quantum and classical computing. In this thesis we provide a compilation 
framework for OPENQASM 3.0, a quantum programming language that supports many classical instructions,
allowing for heterogenous computing. Previous workflows have been provided by this,
notably the QCOR framework, which was a great inspiration for this work. QCOR, and after that this work,
created dialects for lowering of OPENQASM 3.0 to MLIR, apply optimizations and then
execute this intermeidate representation on a QPU paired with a classical accelarator, or just execute it 
on a quantum simulator.
%maybe cite qcor?
In this work we approach this problem similarly, in a couple of aspects, moreover,
we noticed that the work on QCOR was discountinued, and the OPENQASM 3.0 spec was changing
frequently, and we wanted to keep up with the latest changes. 
% TODO: double check if we actually abide to the latest syntax,
% if not just mention that we tried to adhere to the latest available spec.
The difference in our approach are the following:
1- adhere to latest spec as of time of development
2- use latest MLIR version
3- make use of dialects such as the vector dialect, to allow for vectorization of 
operations and perform the much needed speed up of classical operations.
4- Provide an explicit lowering step to restricted gate sate matching quantum gates
available on desired hardware.
5- We did this step as a proof of concept of the possibility to lower a generic quantum dialect
to a restricted quantum dialect that has a specific set of quantum gates. We used quantum gates
supported by the Walter Meiner Institute(WMI)'s quantum computers, as an example.
6- Due to limitations faced by the quantum computing team in WMI, we were not able to execute
the quantum heterogenous generated MLIR code on their hardware.
However, as also another proof of conept, we created a quantum simulator written purely in MLIR
to test out our compilation pipeline, and to showcase the possiblity to write a quantum simulator purely in MLIR,
which could have superior perfromance to call external c++(or other languages) from within our MLIR code.

In the next section we provide a background about the differente topics for this thesis. 
FOllowed by the body were we showcase our approach to this
compilation pipeline, the key components in our codebase, the openqasm operation we 
supported, not all operations were supported but the set of operations can be extended. We also showcase
some example mlir code that was generated at 
different stages of our compilation pipeline.
Finally we show some basic quantum optimization that
we constructed using MLIR's pattern rewriter as part of the compilation passes,
as well as builtin mlir classical optimizations.
Finally, in the last chapeter we showcase our quantum
simulation and present some closing remarks about our works
and the future work that can extend the current one.

% This document has been created in order to show you some of the capabilities 
% of \LaTeX.  A great resource for an introduction to \LaTeX\xspace is Tobias
% Oetiker's ''The Not So Short Introduction to \LaTeXe'' \cite{latex}.  Please
% page through that document
% before starting with your thesis.
% Oh, and let's use the mysterious word \gls{computer} here to give the glossary
% a reason to appear.
% A third useful option to reference stuff besides citing or glossarying (?) 
% is using footnotes. Just like
% this\footnote{Properly formatted clickable URL: \url{https://www.tum.de/}}
% one.
% And: lists! Lists with bullet points are amazing. I mean, just look at this:
% \begin{itemize}
% 	\item list
% 	\item all 
% 	\item the 
% 	\item things!
% \end{itemize}
% % use enumerate for numbers instead of points: 
% % https://en.wikibooks.org/wiki/LaTeX/List_Structures#List_structures
% \par
% Anyways your introduction goes here.


% Below a few \LaTeX examples are included for beginners
% \comment{You can also put comments in the margins for you or your advisor}
% \begin{figure}[ht]
%   \centering
%   \includegraphics[width=5cm]{images/swing_function_plot.png}
%   \caption{$u(x)$}%{Numerically solved solution}
%   \label{fig:swingPlot}
% \end{figure}


% Equations can also be labeled
% \begin{equation}
% 	\pi = \mathrm{e}^{i\cdot\phi}
% 	\label{eq:equation1}
% \end{equation}


% And later referenced. Even in subfigures.
% \begin{figure}[!htb]
%   \centering
%   \begin{subfigure}[b]{0.3\textwidth}
%     \centering
%   \includegraphics[width=\textwidth]{images/CircCenter}
%   \caption{Equation~\ref{eq:equation1}}\label{fig:circcenter}
% \end{subfigure}
% \hfill
%   \begin{subfigure}[b]{0.3\textwidth}
%     \centering
%   \includegraphics[width=\textwidth]{images/GeneralOffset}
%   \label{fig:generaloffset}
%   \caption{Equation~\ref{eq:equation1}}
% \end{subfigure}
% \end{figure}
% \section{Including code}

% Code can be using the package
% \href{https://www.sharelatex.com/learn/Code\_Highlighting\_with\_minted}{Minted}.

% An exaple of which of can be found below (see Source Code~\ref{lst:nice_listing})
% \begin{listing}
% 	%the language syntax can be declared here.
% 	\begin{minted}{python} 
% 	import numpy as np
	
% 	def incmatrix(genl1,genl2):
% 	    m = len(genl1)
% 	    n = len(genl2)
% 	    M = None #to become the incidence matrix
% 	    VT = np.zeros((n*m,1), int)  #dummy variable
	
% 	    #compute the bitwise xor matrix
% 	    M1 = bitxormatrix(genl1)
% 	    M2 = np.triu(bitxormatrix(genl2),1)
	
% 	    for i in range(m-1):
% 	        for j in range(i+1, m):
% 	            [r,c] = np.where(M2 == M1[i,j])
% 	            for k in range(len(r)):
% 	                VT[(i)*n + r[k]] = 1;
% 	                VT[(i)*n + c[k]] = 1;
% 	                VT[(j)*n + r[k]] = 1;
% 	                VT[(j)*n + c[k]] = 1;
	
% 	                if M is None:
% 	                    M = np.copy(VT)
% 	                else:
% 	                    M = np.concatenate((M, VT), 1)
	
% 	                VT = np.zeros((n*m,1), int)
	
% 	    return M
% 	\end{minted}

%   \caption{My nice listing}
%   \label{lst:nice_listing}
% \end{listing}
